\textbf{Berührungspunkte.}

\textbf{Def. 4.2.1 (Berührungspunket).} Es sei $D \subseteq \mathbb K$. Ein Element $x_0 \in \mathbb K$ heißt Berührungspunkt von D, wenn es eine Folge $(x_n)_{n \in \mathbb N}$ mit $x_n \in D$ für alle $n \in \mathbb N$ gibt, so dass $\lim_{n \rightarrow \infty} x_n = x_0$

\textbf{Stetigkeit.} 

\textbf{Def. 4.2.8 (Stetigkeit).} Es sei $D \subseteq \mathbb K$ und $f:D\rightarrow \mathbb K$ eine Funktion.

(i) Die Funktion $f$ heißt stetig in $x_0 \in D$, falls ihr Grenzwert in $x_0$ existiert (also $\lim_{x\rightarrow x_0} f(x) = f(x_0)$).

(iii) Die Funktion $f$ heißt stetig, falls sie in allen Punkten $x_0\in D$ stetig ist.

(iv) Ist $x_0 \in \mathbb K \backslash D$ ein Berührungspunkt von $D$ und existiert der Grenzwert von $f$ in $x_0$, so heißt $f$ stetig ergänzbar in $x_0$. Die Funktion $\overline f:D \cup \{x_0\} \rightarrow \mathbb K$ mit $\overline f(x)=f(x)$ (falls $x\in D$) und $\overline f(x)=\lim_{x'\rightarrow x_0}f(x')$ (falls $x=x_0$) heißt stetige Ergänzung von $f$ in $x_0$.

\textbf{Satz 4.2.16.} Die Exponentialfunktion $\exp :\mathbb C \rightarrow \mathbb C$ ist stetig.

\textbf{Satz 4.2.17 (Stetigkeit von Potenzreihen).} Es sei $\sum_{k=0}^\infty a_k z^k$ eine Potenzreihe mut Konvergenzradius $rho > 0$. Wirbetrachten die zuugehörige komplexe Funktion $f : \{ z \in \mathbb C \vert \vert z \vert < \rho \} \rightarrow \mathbb C\}$ mit $f(z) := \sum_{k=0}^\infty a_k z^k$. Dann ist die Funktion $f$ stetig.

\textbf{Korollar 4.2.18.} Der natürliche Logartihmus $\ln : \mathbb R_{>0} \rightarrow \mathbb R$ ist stetig.

\textbf{Lemma 4.2.19.} Es sei $p:\mathbb R \rightarrow \mathbb R$ eine belibige Polynomfunktion. Dann gilt $\lim_{x\rightarrow \infty} \frac{p(x)}{\exp(x)}=0$. Für beliebiges $n\in \mathbb N$ gilt $\lim_{x\rightarrow \infty} \frac{\ln(x)}{\sqrt[n]{x}} = 0$.

\textbf{Lemma 4.2.20.} Die Exponetialfunktion $\exp : \mathbb C \rightarrow \mathbb C$ besitzt die folgenden Eigenschaften:

(i) $\exp(\overline z) = \overline{\exp(z)}$ für alle $z \in \mathbb C$.

(ii) $\vert \exp(x+iy) \vert = \exp (x)$ für alle $x,y \in \mathbb R$. Insbesontere ist $\vert\exp(iy)\vert =1$ für alle $y \in \mathbb R$.

\textbf{Lemma 4.2.21.} Die Funktion $f:\mathbb R \rightarrow \mathbb C$ mit $f(\phi) := \exp (i\phi)$ ist $2\pi$-periodisch, das heißt $\exp (i(\phi +2\pi)) = \exp(i\phi)$ für alle $\phi \in \mathbb R$.

\textbf{Def. 4.2.22 (Sinus und Cosinus).} Für $\phi \in \mathbb R$ definieren wir $\cos(\phi) := \text{Re}(\exp(i\phi))$ und $\sin(\phi) := \text{Im}(\exp(i\phi))$.

\textbf{Satz 4.2.23 (Eigenschaften von cos und sin).} (i) Cosinus und Sinus sind $2\pi$-periodisch und es gilt für alle $k \in \mathbb Z$:

$\cos (2k\pi) = \sin (2k\pi + \pi / 2)= 1$,

$\cos (k\pi+ \pi /2) = \sin(k\pi) = 0$,

$\cos ((2k+1)\pi) = \sin (2k\pi+3\pi/2) = -1$.

Außerdem ist $\cos (\phi) = \sin(\phi + \pi /2)$ für alle $\phi \in \mathbb R$.

(ii) Cosinus und Sinus sind  stetige Funktionen.

(iii) $(\cos(x))^2 + (\sin (x))^2 = 1$ für alle $x,y \in \mathbb R$.

(iv) $\cos (x+y) = \cos (x) \cos (y) - \sin (x) \sin (y)$ für alle $x,y \in \mathbb R$.

(v) $\sin (x+y) = \sin (x) \cos(y) + \cos (x) \sin (y)$ für alle $x,y \in \mathbb R$.

\textbf{Satz 4.2.24 (Reihenentwicklung von cos und sin).} Die Funktionen $\cos$ und $\sin$ lassen sich wie folgt als Potenzreihen darstellen. Es gilt $\cos (x) = \sum_{k=0}^\infty \frac{(-1)^k}{(2k)!}x^{2k}$ und $\sin (x) = \sum_{k=0}^\infty \frac{(-1)^k}{(2k+1)!}x^{2k+1}$.

\textbf{Def. 4.2.26 (Arcuscosinus und Arcussinus).} Die Umkehrfunktionen von Cosinus und Sinus werden als Arcuscosinus und Arcussinus bezeichnet. Es gilt $\arccos\colon[-1,1]\to[0,\pi]$ und $\arcsin\colon[-1,1]\to \left[-\frac{\pi}{2},\frac{\pi}{2} \right]$.  

\textbf{Def. 4.2.27 (Tangens und Cotangens).} Für $x\in \mathbb R \backslash \{ \pi /2 + k \pi \vert k \in \mathbb Z \}$ ist die Tangens-Funktion definiert als $\tan(x) := \frac{\sin(x)}{\cos(x)}$. Für $x \in \mathbb R \backslash \{ k\pi \vert k \in \mathbb Z \}$ ist die Cotangens-Funktion definiert als $\cot(x) := \frac{\cos(x)}{\sin(x)}$.

\textbf{Nullstellen.}

\textbf{p-q-Formel.} $x_{1} = - \frac{p}{2}+\sqrt{\left(\frac{p}2\right)^2 - q}$

$x_{2} =  \frac{p}{2}-\sqrt{\left(\frac{p}2\right)^2 - q}$

