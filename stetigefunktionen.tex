\textbf{Berührungspunkte.}

\textbf{Def. 4.2.1 (Berührungspunket).} Es sei $D \subseteq \mathbb K$. Ein Element $x_0 \in \mathbb K$ heißt Berührungspunkt von D, wenn es eine Folge $(x_n)_{n \in \mathbb N}$ mit $x_n \in D$ für alle $n \in \mathbb N$ gibt, so dass $\lim_{n \rightarrow \infty} x_n = x_0$

\textbf{Grenzwert.} FIXME

\textbf{Stetigkeit.} FIXME

\textbf{Def. 4.2.8 (Stetigkeit).} FIXME

\textbf{Korollar 4.2.9.} FIXME

\textbf{Satz 4.2.16.} Die Exponentialfunktion $\exp :\mathbb C \rightarrow \mathbb C$ ist stetig.

\textbf{Satz 4.2.17 (Stetigkeit von Potenzreihen).} Es sei $\sum_{k=0}^\infty a_k z^k$ eine Potenzreihe mut Konvergenzradius $rho > 0$. Wirbetrachten die zuugehörige komplexe Funktion $f : \{ z \in \mathbb C \vert \vert z \vert < \rho \} \rightarrow \mathbb C\}$ mit $f(z) := \sum_{k=0}^\infty a_k z^k$. Dann ist die Funktion $f$ stetig.

\textbf{Korollar 4.2.18.} Der natürliche Logartihmus $\ln : \mathbb R_{>0} \rightarrow \mathbb R$ ist stetig.

\textbf{Lemma 4.2.19.} FIXME

\textbf{Lemma 4.2.20.} FIXME

\textbf{Lemma 4.2.21.} FIXME

\textbf{Def. 4.2.22 (Sinus und Cosinus).} FIXME

\textbf{Satz 4.2.23 (Eigenschaften von cos und sin).} FIXME

\textbf{Satz 4.2.24 (Reihenentwicklung von cos und sin).} Die Funktionen $\cos$ und $\sin$ lassen sich wie folgt als Potenzreihen darstellen. Es gilt $\cos (x) = \sum_{k=0}^\infty \frac{(-1)^k}{(2k)!}x^{2k}$ und $\sin (x) = \sum_{k=0}^\infty \frac{(-1)^k}{(2k+1)!}x^{2k+1}$.

\textbf{Def. 4.2.26 (Arcuscosinus und Arcussinus).} Die Umkehrfunktionen von Cosinus und Sinus werden als Arcuscosinus und Arcussinus bezeichnet. Es gilt $\arccos\colon[-1,1]\to[0,\pi]$ und $\arcsin\colon[-1,1]\to \left[-\frac{\pi}{2},\frac{\pi}{2} \right]$.  

\textbf{Def. 4.2.27 (Tangens und Cotangens).} Für $x\in \mathbb R \backslash \{ \pi /2 + k \pi \vert k \in \mathbb Z \}$ ist die Tangens-Funktion definiert als $\tan(x) := \frac{\sin(x)}{\cos(x)}$. Für $x \in \mathbb R \backslash \{ k\pi \vert k \in \mathbb Z \}$ ist die Cotangens-Funktion definiert als $\cot(x) := \frac{\cos(x)}{\sin(x)}$.

