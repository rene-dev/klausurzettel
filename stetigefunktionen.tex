\textbf{Berührungspunkte.}

\textbf{Def. 4.2.1 (Berührungspunket).} Es sei $D \subseteq \mathbb K$. Ein Element $x_0 \in \mathbb K$ heißt Berührungspunkt von D, wenn es eine Folge $(x_n)_{n \in \mathbb N}$ mit $x_n \in D$ für alle $n \in \mathbb N$ gibt, so dass $\lim_{n \rightarrow \infty} x_n = x_0$

\textbf{Grenzwert.} FIXME

\textbf{Stetigkeit.} FIXME

\textbf{Def. 4.2.8 (Stetigkeit).} FIXME

\textbf{Korollar 4.2.9.} FIXME

\textbf{Lemma 4.2.20.} FIXME

\textbf{Lemma 4.2.21.} FIXME

\textbf{Def. 4.2.22 (Sinus und Cosinus).} FIXME

\textbf{Satz 4.2.23 (Eigenschaften von cos und sin).} FIXME

\textbf{Satz 4.2.24 (Reihenentwicklung von cos und sin).} FIXME

\textbf{Def. 4.2.26 (Arcuscosinus und Arcussinus).} FIXME

\textbf{Def. 4.2.27 (Tangens und Cotangens).} FIXME

