\textbf{Methoden der Beweisführung.}

\textbf{Direkter Beweis.} $(A \Rightarrow B)$

\textbf{Beweis durch Kontraposition.} $(A \Rightarrow B) \Leftrightarrow (\neg B \Rightarrow \neg A)$

\textbf{Beweis durch Widerspruch.} $(A \Rightarrow B ) \Leftrightarrow (\neg (A \wedge \neg B))$

\textbf{Vollständige Induktion.}

$(A(0) \wedge ( \forall  n \in \mathbb N : A(n) \Rightarrow A(n+1))) \Rightarrow \left(\forall n \in \mathbb N \colon A(n)\right)$

Induktionsanfang: $A(0)$ ist wahr

Induktionsvoraussetzung: Es gelte A(n) für ein belibiges aber festes $n \in \mathbb N$

Induktionsschluss: $n \mapsto n+1$, Gilt $A(n+1)$, dann gilt  $\forall n \in \mathbb{N}$ $A(n)$

