\textbf{Kombinatorik.}

\textbf{Def. 5.1.4 (Binomialkoeffizient).} Für $n \in \mathbb N_0$ und $k \in \mathbb N$ wird der Binomialkoeffizient als $\binom{n}{k} := ( \frac{n}{1} \cdot \frac{n-1}{2} \cdot \cdots \cdot \frac{n-k+1}{k}$ definiert. Für $k=0$ setzt man $\binom{n}{0} = 1$ für alle $n \in \mathbb N_0$. 

\textbf{Bemerkung.} (i) Für $n,k \in \mathbb N_0$ mit $n \geq k$ ist $\binom{n}{k} = \frac{n!}{k!\cdot (n-k)!}$.

(ii) Für $n<k$ gilt $\binom{n}{k} = 0$.

(iii) Für $n \in \mathbb N_0$ gilt $\binom{n}{0} = \binom{n}{n} = 1$.

(iv) Für $0 \leq k \leq n$ gilt wegen (i) die Symmetrieeigenschaft $\binom{n}{k} = \binom{n}{n-k}$.

(v) Die Definition von $\binom{x}{k}$ ergibt auch für beliebiges $x\in \mathbb R$ Sinn. Wenn $x$ keine ganze Zahl ist, ergibt sich auch für $x<k$ ein von Null verschiedener Wert.