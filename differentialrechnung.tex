Differentialrechnung FIXME



Ableitungregeln FIXME



\textbf{Satz 4.3.20 (L'Hôpital).} Es seinen $a,b\in\mathbb{R}$ mit
$a<b$ und $f,g:(a,b]\rightarrow\mathbb{R}$ zwei differenzierbare
Funktionen mit $g(x)\not=0$ und $g'(x)\not=0$ für alle $x\in(a,b]$.

\textbf{(i)} Gilt $\lim_{x\rightarrow a}f(x)=\lim_{x\rightarrow a}g(x)=0$
und existiiert der (uneigentliche) Grenzwert $\lim_{x\rightarrow a}(f'(x)/g'(x))\in\mathbb{R}\cup\{-\infty,\infty\}$,
so gilt $\lim_{x\rightarrow a}\frac{f(x)}{g(x)}=\lim_{x\rightarrow a}\frac{f'(x)}{g'(x)}$.

\textbf{(ii) }Gilt $\lim_{x\rightarrow a}f(x)=\pm\infty$ und $\lim_{x\rightarrow a}g(x)=\pm\infty$
und exitstert der (uneigentliche) Grenzwert $\lim_{x\rightarrow a}(f'(x)/g'(x))\in\mathbb{R}\cup\{-\infty,\infty\},$
so gilt $\lim_{x\rightarrow a}\frac{f(x)}{g(x)}=\lim_{x\rightarrow a}\frac{f'(x)}{g'(x)}$.

Taylorreihen FIXME

\textbf{Def. 4.3.21 (Taylorpolynom).} Es sei $D\subseteq\mathbb{R},x_{0}\in D$
und $f:D\rightarrow\mathbb{R}$ eine Funktion, die an der Stelle $x_{0}$
mindestens $n$-mal differenzierbar ist für ein $n\in\mathbb{N}.$
Dann heißt das Polynom $T_{f,n}(x,x_{0}):=\sum_{i=0}^{n}\frac{f^{(i)}(x_{0})}{i!}(x-x_{0})^{i}$
$n$-tes Taylorpolynom von f mit Entwicklungspunkt $x_{0}$.

\textbf{Satz 4.3.23 (Satz v. Taylor).} Es sei $n\in\mathbb{N}$, $a<b$
und $f:[a,b]\rightarrow\mathbb{R}$ eine n-mal stetig differenzierbare
Funktion, die auf dem offenen Intervall $(a,b)$ mindestens $(n+1)$-mal
differenzierbar ist. Dann gibt es zu jedem $x\in(a,b]$ ein $y\in(a,x)$
mit $f(x)=T_{f,n}(x,a)+\frac{f^{(n+1)}(y)}{(n+1)!}(x-a)^{n+1}$.

\textbf{Def. 4.3.24 (Taylor-Reihen).} Es seien $a,b\in\mathbb{R}$
mit $a<b$ und $f:(a,b)\rightarrow\mathbb{R}$ eine unendlich oft
differenzierbare Funktion. Dann heißt für $x_{0}\in(a,b)$ die Reihe
$T_{f}(x,x_{0}):=\sum_{k=0}^{\infty}\frac{f^{(k)}(x_{0})}{k!}(x-x_{0})^{k}$
die Taylor-Reihe von $f$ mit Entwicklungspunkt $x_{0}$.

Funktionen mehrerer Verenderlicher FIXME

\textbf{Def. 4.3.27 (Partielle Ableitung).} FIXME

