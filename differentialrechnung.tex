\textbf{Differentialrechnung.}

\textbf{Def. 4.3.1 (Sekante, Steigung).} Es sei $D \subseteq \mathbb R$ und $f : D \rightarrow \mathbb R$ eine reelle Funktion. Sind $x_0$, $x \in D$ mit $x_0 \neq x$, dann heißt die eindeutige Gerade durch die Punkte $(x_0,f(x_0))$ und $(x,f(x))$ Sekante des Graphen der Fnuktion $f$. Die Steigung der Sekante ist durch den Ausdruck $\frac{f(x)-f(x_0)}{x-x_0}$ gegeben. Dieser Ausdruck wird auch Differenzenquotient genannt.

\textbf{Def. 4.3.2 (Differenzierbarkeit, Ableitung).} Es sei $D \subseteq \mathbb R$ und $f: D \rightarrow \mathbb R$ eine reelle Funktion.

(i) Ist $x_0 \in D$ und existiert der Grenzwert $\lim_{x \rightarrow x_0} \frac{f(x)-f(x_0)}{x-x_0}$, so sagt man, dass $f$ in $x_0$ differenzierbar ist.

(ii) Eine Funktion, die in jedem Punkt differenzierbar ist, nennt man Differenzierbar.

\textbf{Satz 4.3.4.} Es sei $D \subseteq \mathbb R$ und $f : D \rightarrow \mathbb K$ eine Funktion. Ist $x_0 \in D$ und ist $f$ in $x_0$ differenzierbar, so ist $f$ in $x_0$ auch stetig.

\textbf{Satz 4.3.5.} Die reelle Exponentialfunktion $\exp : \mathbb R \rightarrow \mathbb R$ ist differenzierbar und es gilt $\exp' = \exp$.

\textbf{Korollar 4.3.6.} Für $c \in \mathbb C$ ist die Funktion $f:\mathbb R \rightarrow \mathbb C$ mit $f(x) := \exp(c\cdot x)$ differenzierbar mit Ableitung $f'(x) = c \cdot f(x) = c \cdot \exp (c \cdot x)$.

\textbf{Satz 4.3.8.} Es sei $D\subseteq \mathbb R$, $x_0 \in \mathbb D$ und $f,g : D \rightarrow \mathbb K$ zwei Funktionen, die un $x_0$ differenzierbar sind. Weiter seien $\lambda , \mu \in \mathbb K$. Dann gilt:

(i) Die Funkion $\lambda \cdot f + \mu \cdot g : D \rightarrow \mathbb K$ ist in $x_0$ differenzierbar mit $(\lambda \cdot f + \mu \cdot g)' (x_0) = \lambda \cdot f'(x_0) + \mu \cdot g' (x_0)$ („Linearität der Ableitung“).

(ii) Die Funktionen $f\cdot g : D \rightarrow \mathbb K$ ist in $x_0$ differenzierbar mit $(f \cdot g)'(x_0) = f'(x_0)\cdot g(x_0) + f(x_0) \cdot g'(x_0)$ („Leibniz’sche Produktregel“).

(iii) Ist $g(x_0) \neq 0$, so ist die Funktion $f/g$ in $x_0$ defferenzierbar mit $\left(\frac{f}{g}\right)'(x_0) = \frac{f'(x_0)\cdot g(x_0) - f(x_0) \cdot g'(x_0)}{g(x_0)^2}$ („Quotientenregel“).

(iv) Ist $D' \subseteq \mathbb R$, $y_0 \in D'$ und $h:D' \rightarrow D$ eine Funktion mit $h(y_0) = x_0$, die in $y_0$ differenzierbar ist. Dann ist die Funktion $f \circ h : D' \rightarrow \mathbb K$ in $y_0$ differenzierbar mit $(f \circ h)'(y_0) = f'(h(y_0)) \cdot h'(y_0)$ („Kettenregel“).

\textbf{Satz 4.3.9 (Ableitung der Umkehrfunktion).} FIXME $(f^{-1})' (f(x_0)) = \frac{1}{f'(x_0)}$

\textbf{Satz 4.3.13 (Mittelwertsatz der Differentialrechnung).} FIXME 

\textbf{Korollar 4.3.14 (Satz von Rolle).} FIXME

\textbf{4.3.19 (Verallgemeinerter Mittelwertsatz).} FIXME

\textbf{Satz 4.3.20 (L'Hôpital).} Es seinen $a,b\in\mathbb{R}$ mit
$a<b$ und $f,g:(a,b]\rightarrow\mathbb{R}$ zwei differenzierbare
Funktionen mit $g(x)\not=0$ und $g'(x)\not=0$ für alle $x\in(a,b]$.

\textbf{(i)} Gilt $\lim_{x\rightarrow a}f(x)=\lim_{x\rightarrow a}g(x)=0$
und existiiert der (uneigentliche) Grenzwert $\lim_{x\rightarrow a}(f'(x)/g'(x))\in\mathbb{R}\cup\{-\infty,\infty\}$,
so gilt $\lim_{x\rightarrow a}\frac{f(x)}{g(x)}=\lim_{x\rightarrow a}\frac{f'(x)}{g'(x)}$.

\textbf{(ii)} Gilt $\lim_{x\rightarrow a}f(x)=\pm\infty$ und $\lim_{x\rightarrow a}g(x)=\pm\infty$
und exitstert der (uneigentliche) Grenzwert $\lim_{x\rightarrow a}(f'(x)/g'(x))\in\mathbb{R}\cup\{-\infty,\infty\},$
so gilt $\lim_{x\rightarrow a}\frac{f(x)}{g(x)}=\lim_{x\rightarrow a}\frac{f'(x)}{g'(x)}$.

\textbf{Taylorreihen.}

\textbf{Def. 4.3.21 (Taylorpolynom).} Es sei $D\subseteq\mathbb{R},x_{0}\in D$
und $f:D\rightarrow\mathbb{R}$ eine Funktion, die an der Stelle $x_{0}$
mindestens $n$-mal differenzierbar ist für ein $n\in\mathbb{N}.$
Dann heißt das Polynom $T_{f,n}(x,x_{0}):=\sum_{i=0}^{n}\frac{f^{(i)}(x_{0})}{i!}(x-x_{0})^{i}$
$n$-tes Taylorpolynom von f mit Entwicklungspunkt $x_{0}$.

\textbf{Satz 4.3.23 (Satz v. Taylor).} Es sei $n\in\mathbb{N}$, $a<b$
und $f:[a,b]\rightarrow\mathbb{R}$ eine n-mal stetig differenzierbare
Funktion, die auf dem offenen Intervall $(a,b)$ mindestens $(n+1)$-mal
differenzierbar ist. Dann gibt es zu jedem $x\in(a,b]$ ein $y\in(a,x)$
mit $f(x)=T_{f,n}(x,a)+\frac{f^{(n+1)}(y)}{(n+1)!}(x-a)^{n+1}$.


\textbf{Def. 4.3.24 (Taylor-Reihen).} Es seien $a,b\in\mathbb{R}$
mit $a<b$ und $f:(a,b)\rightarrow\mathbb{R}$ eine unendlich oft
differenzierbare Funktion. Dann heißt für $x_{0}\in(a,b)$ die Reihe
$T_{f}(x,x_{0}):=\sum_{k=0}^{\infty}\frac{f^{(k)}(x_{0})}{k!}(x-x_{0})^{k}$
die Taylor-Reihe von $f$ mit Entwicklungspunkt $x_{0}$.

\textbf{Funktionen mehrerer Verenderlicher.}

\textbf{Def. 4.3.27 (Partielle Ableitung).} FIXME

\textbf{Def. 4.3.28 (Lokale Extrema).} FIXME

\textbf{Lemma 4.3.29} FIXME

\textbf{Def. 4.3.30 (Hesse-Matrix).} FIXME

\textbf{Korrollar 4.3.32.} FIXME

\textbf{Def. 4.4.33 (Determinate).} FIXME

\textbf{Satz 4.3.34} FIXME

\textbf{Ableitungen} FIXME

