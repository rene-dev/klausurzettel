%% LyX 2.0.0 created this file.  For more info, see http://www.lyx.org/.
%% Do not edit unless you really know what you are doing.
\documentclass[ngerman]{article}

\usepackage[T1]{fontenc}
\usepackage[utf8x]{inputenc}
\usepackage[ngerman]{babel}
\usepackage[a4paper]{geometry}
\usepackage{amssymb}
\usepackage{amsmath}
\usepackage{multicol}

\pagestyle{empty}
\geometry{verbose,tmargin=5mm,bmargin=5mm,lmargin=5mm,rmargin=5mm,headheight=0mm,headsep=0mm,footskip=0mm}
\setlength{\parindent}{0mm}
\makeatletter
\makeatother

\begin{document}
\begin{multicols}{4}
\begin{tiny}

\textbf{Leonhard Küper}

\textbf{Potenzgesetze.} 
$a^{0}=1$

$a^{-r}=\frac{1}{a^{r}}$

$a^{\frac{m}{n}}=\sqrt[n]{a^{m}}=\left(\sqrt[n]{a}\right)^{m}$

$a^{r+s}=a^{r}\cdot a^{s}$

$a^{r-s}=\frac{a^{r}}{a^{s}}$

$(a\cdot b)^{r}=a^{r}\cdot b^{r}$

$\left(\frac{a}{b}\right)^{r}=\frac{a^{r}}{b^{r}}$; $\left(a^{r}\right)^{s}=a^{r\cdot s}$

\textbf{Komplexe Zahlen.} $(a+b\mathrm i)+(c+d\mathrm i)=(a+c)+(b+d)\mathrm i$ ; $(a + b \mathrm i) - (c + d \mathrm i) = (a - c) + (b - d) \mathrm i$

$(a+b\mathrm{i})\cdot(c+d\mathrm{i})=(ac-bd) + (ad+bc)\cdot\mathrm i$

$\frac{a+b\,\mathrm i}{c+d\,\mathrm i} = \frac{(a+b\,\mathrm i)(c-d\,\mathrm i)}{(c+d\,\mathrm i)(c-d\,\mathrm i)} = \frac{ac+bd}{c^2+d^2}+\frac{bc-ad}{c^2+d^2}\cdot\mathrm i$

\textbf{Methoden der Beweisführung.}

\textbf{Direkter Beweis.} $(A \Rightarrow B)$

\textbf{Beweis durch Kontraposition.} $(A \Rightarrow B) \Leftrightarrow (\neg B \Rightarrow \neg A)$

\textbf{Beweis durch Widerspruch.} $(A \Rightarrow B ) \Leftrightarrow (\neg (A \wedge \neg B))$

\textbf{Vollständige Induktion.}

$(A(0) \wedge ( \forall  n \in \mathbb N : A(n) \Rightarrow A(n+1))) \Rightarrow \left(\forall n \in \mathbb N \colon A(n)\right)$

Induktionsanfang: $A(0)$ ist wahr ; Induktionsvoraussetzung: Es gelte A(n) für ein belibiges aber festes $n \in \mathbb N$ ; Induktionsschluss: $n \mapsto n+1$, Gilt $A(n+1)$, dann gilt  $\forall n \in \mathbb{N}$ $A(n)$

\textbf{Def. 1.5.8 (Max., Min., Sup., \& Inf.).} Es sei $(M,\preceq)$ eine Halbordnung und $X$ Teilmenge von $M$. Ein Element $a\in X$ heißt \{minimales / maximales\} Element in $X$, wenn für kein $x\in X$ gilt \{ $x\prec a$ / $x \succ a$ \}.

Ein Element $a \in X$ \{ kleinstes / größtes \} Element in $X$, wenn für alle $x \in X$ gilt \{ $a \preceq x$ / $a \succeq$ \}.

Ein Element $b \in X$ \{ untere / obere \} Schranke von $X$, wenn für alle $x \in X$ gilt \{ $b \preceq x$ / $b \succeq$ \}.

Ein Element $ b \in M$ heißt \{ Infimum / Supremum \} von $X$, wenn $b$ \{ größtes / kleinstes \} Element der Menge aller \{ unteren / oberen \} Schranken von $X$ ist.

\textbf{Die reellen Zahlen.}

\textbf{Def. 4.1.1 (Anordnungsaxiome).} Ein Körper $K$ heißt angeordnet, wenn es in ihm eine Teilmenge $P$ (den Positivbereich gibt, der die folgenden Eigenschaften besitzt:

(i) K ist disjunkte Vereinigung der Mengen $P$, $\{ 0 \}$ und $-P := \{ x \in K \vert -x \in P \}$

(ii) Für $x,y \in P$ gilt: $x + y \in P$ und $x \cdot y \in P$

\textbf{Folgen.}

\textbf{Def. 4.1.9 (Konv. v. Folgen).} Es sei $(a_n)_{n \in \mathbb N}$ eine Folge von komplexen Zahlen $a_n \in \mathbb C$. Die Folge konvergiert gegen den Grenzwert $a \in \mathbb C$ ($\lim_{n \rightarrow \infty} a_n = a$) wenn gilt: $\forall \epsilon > 0 \exists n_0 \in \mathbb N \forall n > n_0 : \vert a_n - a \vert < \epsilon$. 

\textbf{Satz 4.1.10 (Eindeutigkeit des Grenzwerts).} Es sei $(a_n)_{n \in \mathbb N}$ eine konvergente Folge mit $a_n \in \mathbb{C}$. Dann ist der Grenzwert a der Folge eindeutig bestimmt.

\textbf{Satz 4.1.14 (Vergleichskriterium).} Es seien $(a_n)$, $(b_n)$ und $(c_n)$ reelle Zahlenfolgen mit $a_n \leq b_n \leq c_n$ für alle $n \in \mathbb N$.

(i) Ist $b \in \mathbb R$ mit $\lim_{n \rightarrow \infty} a_n\ = \lim_{n \rightarrow \infty} c_n = b$, so ist $\lim_{n \rightarrow \infty} b_n = b$.

(ii) Ist $\lim_{n \rightarrow \infty} a_n = \infty$, so ist auch $\lim_{n \rightarrow \infty} b_n = \infty$.

(iii) Ist $\lim_{n \rightarrow \infty} c_n = -\infty$, so ist auch $\lim_{n \rightarrow \infty} b_n = -\infty$.

\textbf{Satz 4.1.17 (Verknüpfungen konvergenter Folgen).} Es seien $(a_n)_{n\in \mathbb N}$ und $(b_n)_{n \in \mathbb N}$ konvergente komplexe Zahlenfolgen, $a := \lim_{n \rightarrow \infty} a_n$ und $b := \lim_{n \rightarrow \infty} b_n$. Dann gilt:

(i) Die Folge $(a_n \pm b_n)$ ist konvergen mit $\lim_{n \rightarrow \infty} (a_n \pm b_n) = a \pm b$.

(ii) Die Folge $(a_n \cdot b_n)$ ist konvergent mit $\lim_{n \rightarrow \infty} (a_n \cdot b_n) = a \cdot b$.

(iii) Ist $c \in \mathbb C$, so ist die Folge $(c \cdot a_n)$ konvergent mit $\lim_{n \rightarrow \infty} (c \cdot a_n ) = a \cdot b$.

(iv) Ist $b \neq 0$, so gibt es ein $n_0 \in \mathbb N$, so dass $b_n \neq 0$ für alle $n \geq n_0$. Dann sind auch die Folgen $(\frac{1}{b_n})$ und $(\frac{a_n}{b_n})$ konvergent mit $\lim_{n\rightarrow \infty } (\frac{1}{b_n}) = \frac{1}{b}$ und $\lim_{n\rightarrow \infty } (\frac{a_n}{b_n}) = \frac{a}{b}$.

(v) Die Folge $(\vert a_n \vert )_{n \in \mathbb N}$ ist konvergent mit $\lim_{n \rightarrow \infty } \vert a_n \vert = \vert a \vert$.

(vi) Die komplexe Folge $(a_n)_{n \in \mathbb N}$ ist genau dann konvergent, wenn die beiden reellen Zahlenfolgen $(\operatorname{Re} (a_n))_{n \in \mathbb N}$ und $(\operatorname{Im} (a_n))_{n \in \mathbb N}$ konvergieren. In diesem Fall gilt $\lim_{n \rightarrow \infty} a_n = \lim_{n \rightarrow \infty} \operatorname{Re} (a_n) + i \cdot \lim_{n \rightarrow \infty} \operatorname{Im} (a_n)$.

\textbf{Wichtige Folgen.}

$\lim_{n \rightarrow \infty} \frac{1}{n} = 0$

$\lim_{n \rightarrow \infty} \frac{n}{1} = \infty$

$\lim_{n \rightarrow \infty} \sqrt[n]{n} = 1$

$\lim_{n \rightarrow \infty} \frac{n}{n+1} = 1$

\textbf{Reihen.}

\textbf{Satz 4.1.27 (Majoranten- und Minorantenkriterium).} Es seinen $(a_n)_{n \in \mathbb N}$ und $(b_n)_{n \in \mathbb N}$ Folgen komplexer Zahlen.

(i) Ist $\sum_{k=1}^\infty b_k$ absolut konvergent und gibt es eine reelle Zahl $c > 0$ mit $\vert a_n\vert \leq c \cdot \vert b_n \vert$ für alle $n \in \mathbb N$, dann ist auch $\sum_{k=1}^\infty a_k$ absolut konvergent.

(ii) Ist $\sum_{k=1}^\infty b_k$ nicht absolut konvergent und gibt es eine reelle Zahl $c > 0$ mit $\vert a_n\vert \geq c \cdot \vert b_n \vert$ für alle $n \in \mathbb N$, dann ist auch $\sum_{k=1}^\infty a_k$ nicht absolut konvergent.

\textbf{Satz 4.1.28 (Quotientenkriterium).} Es sei $(a_n)_{n\in \mathbb N}$ eine Folge komplexer Zahlen.

(i) Gibt es ein $q < 1$, so dass $\frac{a_{n+1}}{a_n} \leq q$ für alle $n \geq n_0 \in \mathbb N$, dann ist die zugehörige Reihe $\sum_{k=1}^\infty a_k$ absolut konvergent.

FIXME

\textbf{Satz 4.1.29 (Wurzelkriterium).} Es sei $(a_n)_{n \in \mathbb N}$ eine Folge komplexer Zahlen.

(i) Gibt es ein FIXME

\textbf{Satz 4.1.31 (Leibnitz-Kriterium).} Es sei $\sum_{k=1}^\infty a_k$ eine alternierende Reihe. Ist die Folge $(\vert a_n \vert )_{n \in \mathbb N}$ eine monoton fallende Nullfolge, dann konvergiert die Reihe $\sum_{k=1}^\infty a_k$.

\textbf{Wichtige Reihen.}

Geometrische Reihe: $\sum_{k=0}^\infty q^k = \frac{1}{1-q}$ für $\vert q \vert < 1$

Harmonische Reihe: $\sum_{k=0}^\infty \frac{1}{k} = \infty$

$\sum_{k=0}^\infty \frac{1}{k^2} = \frac{\pi^2}{6}$

$\sum_{k=0}^\infty \frac{k!}{k^k} =$ FIXME


\textbf{Potenzreihen.}

\textbf{Def. 4.1.35 (Potenzreihen).} Es sei $(a_n)_{n \in \mathbb N_0}$ eine Folge komplexer Zahlen. Dann ist für belibige $z \in \mathbb C$ $\sum_{k=0}^\infty a_k z^k$ eine Potenzreihe.

\textbf{Def. 4.1.37 (Konvergenzradius).} FIXME

\textbf{Satz 4.1.38 (Konvergenz von Potenzreihen).} FIXME

\textbf{Satz 4.1.39} Es sei $(a_n)_{n \in \mathbb N_0}$ eine Folge komplexer Zahlen und $\sum_{k=0}^\infty a_k z^k$ die zugehörige Potenzreihe mit Konvergenzradius $\rho$. Für $n \in \mathbb N_0$ sei $b_n := \sqrt[n]{\vert a_n \vert}$. Dann gilt:

(i) Geht die Folge $(b_n)_{n \in \mathbb N_0}$ gegen unendlich, so ist $\rho = 0$.

(ii) Konvergiert $(b_n)_{n \in \mathbb N_0}$ gegen den Grenzwert $b > 0$, so ist $\rho = \frac{1}{b}$.

(iii) Ist $(b_n)_{n \in \mathbb N_0}$ eine Nullfolge, so ist $\rho = \infty$.

\textbf{Satz 4.1.40} Es sei $(a_n)_{n \in \mathbb N_0}$ eine Folge komplexer Zahlen mit $a_n \neq 0$ für alle $n \in \mathbb N_0$ und $\sum_{k=0}^\infty a_k z^k$ die zugehörige Potenzreihe mit Konvergenzradius $\rho$. Für $n \in \mathbb N_0$ sei $b_n := \frac{\vert a_{n+1} \vert }{\vert a_n \vert}$. Dann gilt:

(i) Geht die Folge $(b_n)_{n \in \mathbb N_0}$ gegen unendlich, so ist $\rho = 0$.

(ii) Konvergiert $(b_n)_{n \in \mathbb N_0}$ gegen den Grenzwert $b > 0$, so ist $\rho = \frac{1}{b}$.

(iii) Ist $(b_n)_{n \in \mathbb N_0}$ eine Nullfolge, so ist $\rho = \infty$.

\textbf{Wichtige Potenzreihen.}

\textbf{Exponential- und Fogarithmus-Funktion.}

\textbf{Satz 4.1.43 (Euler’sche Zahl e).} Die Folge $(a_n)_{n \in \mathbb N}$ mit $a_n := (1+\frac{1}{n})^n$ konvergiert. Der Grenzwert wird mit e bezeichnet und Euler’sche Zahl genannt.

\textbf{Lemma 4.1.44} Die Reihe $\sum_{k=0}^\infty \frac{1}{k!}$ konvergiert gegen den Grenzwert e.

\textbf{Def. 4.1.45 (Exponentialfunktion).} Die Exponentialfunktion $\exp : \mathbb C \rightarrow \mathbb C$ ist definiert durch $\exp (z) :=  \sum_{k=0}^\infty \frac{z^k}{k!}$ für alle $z \in \mathbb C$.

\textbf{Bemerkung.} $e^z = \exp (z)$ für alle $z \in \mathbb C$

\textbf{Def. 4.1.50 (Natürlicher Logarithmus).} FIXME

\textbf{Def. 4.1.52 (Exponetialfunktion zur Basis a).} FIXME

\textbf{Lemma 4.1.55.} FIXME

\textbf{Berührungspunkte.}

\textbf{4.2.1 (Berührungspunket).} FIXME

\textbf{Grenzwert.} FIXME

\textbf{Stetigkeit.} FIXME

\textbf{Def. 4.2.8 (Stetigkeit).} FIXME

\textbf{Korollar 4.2.9.} FIXME

\textbf{Lemma 4.2.20.} FIXME

\textbf{Lemma 4.2.21.} FIXME

\textbf{Def. 4.2.22 (Sinus und Cosinus).} FIXME

\textbf{Satz 4.2.23 (Eigenschaften von cos und sin).} FIXME

\textbf{Satz 4.2.24 (Reihenentwicklung von cos und sin).} FIXME

\textbf{Def. 4.2.26 (Arcuscosinus und Arcussinus).} FIXME

\textbf{Def. 4.2.27 (Tangens und Cotangens).} FIXME

Differentialrechnung FIXME



Ableitungregeln FIXME



\textbf{Satz 4.3.20 (L'Hôpital).} Es seinen $a,b\in\mathbb{R}$ mit
$a<b$ und $f,g:(a,b]\rightarrow\mathbb{R}$ zwei differenzierbare
Funktionen mit $g(x)\not=0$ und $g'(x)\not=0$ für alle $x\in(a,b]$.

\textbf{(i)} Gilt $\lim_{x\rightarrow a}f(x)=\lim_{x\rightarrow a}g(x)=0$
und existiiert der (uneigentliche) Grenzwert $\lim_{x\rightarrow a}(f'(x)/g'(x))\in\mathbb{R}\cup\{-\infty,\infty\}$,
so gilt $\lim_{x\rightarrow a}\frac{f(x)}{g(x)}=\lim_{x\rightarrow a}\frac{f'(x)}{g'(x)}$.

\textbf{(ii) }Gilt $\lim_{x\rightarrow a}f(x)=\pm\infty$ und $\lim_{x\rightarrow a}g(x)=\pm\infty$
und exitstert der (uneigentliche) Grenzwert $\lim_{x\rightarrow a}(f'(x)/g'(x))\in\mathbb{R}\cup\{-\infty,\infty\},$
so gilt $\lim_{x\rightarrow a}\frac{f(x)}{g(x)}=\lim_{x\rightarrow a}\frac{f'(x)}{g'(x)}$.

Taylorreihen FIXME

\textbf{Def. 4.3.21 (Taylorpolynom).} Es sei $D\subseteq\mathbb{R},x_{0}\in D$
und $f:D\rightarrow\mathbb{R}$ eine Funktion, die an der Stelle $x_{0}$
mindestens $n$-mal differenzierbar ist für ein $n\in\mathbb{N}.$
Dann heißt das Polynom $T_{f,n}(x,x_{0}):=\sum_{i=0}^{n}\frac{f^{(i)}(x_{0})}{i!}(x-x_{0})^{i}$
$n$-tes Taylorpolynom von f mit Entwicklungspunkt $x_{0}$.

\textbf{Satz 4.3.23 (Satz v. Taylor).} Es sei $n\in\mathbb{N}$, $a<b$
und $f:[a,b]\rightarrow\mathbb{R}$ eine n-mal stetig differenzierbare
Funktion, die auf dem offenen Intervall $(a,b)$ mindestens $(n+1)$-mal
differenzierbar ist. Dann gibt es zu jedem $x\in(a,b]$ ein $y\in(a,x)$
mit $f(x)=T_{f,n}(x,a)+\frac{f^{(n+1)}(y)}{(n+1)!}(x-a)^{n+1}$.

\textbf{Def. 4.3.24 (Taylor-Reihen).} Es seien $a,b\in\mathbb{R}$
mit $a<b$ und $f:(a,b)\rightarrow\mathbb{R}$ eine unendlich oft
differenzierbare Funktion. Dann heißt für $x_{0}\in(a,b)$ die Reihe
$T_{f}(x,x_{0}):=\sum_{k=0}^{\infty}\frac{f^{(k)}(x_{0})}{k!}(x-x_{0})^{k}$
die Taylor-Reihe von $f$ mit Entwicklungspunkt $x_{0}$.

Funktionen mehrerer Verenderlicher FIXME

\textbf{Def. 4.3.27 (Partielle Ableitung).} FIXME

Integralrechnung FIXME

Treppenfunktionen FIXME

Integrationsregeln FIXME

Uneigentliche Integrale FIXME

Stammmfunktion (substitution, partiellee integration, uneigentliche integrale)

\textbf{Def. 5.1.4 Binomialkoeffizient} FIXME

\textbf{Def 5.3.1 (Endlicher Graph).} FIXME 

\textbf{Def. 5.3.2 (Vollständiger Graph).} FIXME

\textbf{Def. 5.3.3 (Knotengrad).} FIXME

\textbf{Def. 5.3.5 (Kantenfogle, Wege, Kreise).} FIXME

\textbf{Def. 5.3.8 (Teilgraph).} FIXME

\textbf{Def. 5.3.9 (Zusammenhängender Graph).} FIXME

\textbf{Def. 5.3.10 (Zusammenhangskomponenten).} FIXME

\textbf{Def. 5.3.11 (Euler\textquoteright{}scher Graph, Euler-Tour).} FIXME

\textbf{Satz 5.3.12 (Satz von Euler).} FIXME

\textbf{Satz 5.3.17} Ein Baum $G=(V,E)$ besitzt genau eine Kante
weniger als Knoten, das heißt $\vert E\vert=\vert V\vert-1$

\textbf{Def. 5.3.19 (Zyklomatische Zahl).} FIXME

\end{tiny}
\end{multicols}
\end{document}
