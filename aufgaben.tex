%Erzeugt mit dem LaTeX-Generator: http://latex.sehnot.de

%Schriftgröße, Layout, Papierformat, Art des Dokumentes
\documentclass[11pt,oneside,a4paper]{scrartcl}

%Einstellungen der Seitenränder
\usepackage[left=2cm,right=2cm,top=2cm,bottom=2cm,includeheadfoot]{geometry}

%neue Rechtschreibung
\usepackage{ngerman}

%Umlaute ermöglichen
%\usepackage[latin1]{inputenc}
%\usepackage[applemac]{inputenc}
\usepackage[utf8]{inputenc}


%Komische Zeichen
\usepackage{amssymb}
\usepackage{amsmath}
\usepackage{listings}

%Grafik
\usepackage{graphics}
\usepackage{graphicx}
\usepackage{tikz}


%Kopf- und Fußzeile
\usepackage{fancyhdr}
\pagestyle{fancy}
\fancyhf{}

%Kopfzeile links bzw. innen
%\fancyhead[R]{\today}
%Kopfzeile mittig
\fancyhead[L]{Mafi2 Klausuraufgaben}
%Linie oben
\renewcommand{\headrulewidth}{0.5pt}

%Fußzeile rechts bzw. außen
\fancyfoot[R]{\thepage}
%Linie unten
\renewcommand{\footrulewidth}{0pt}
\newcommand{\qed}{\begin{flushright}$^\blacksquare$\end{flushright}}

%\documentclass{article}
\usepackage{hyperref}
\usepackage{amsmath}
%\usepackage[applemac]{inputenc}
%\usepackage[ngerman]{babel}

%\usepackage{fancyhdr}
%\pagestyle{plain}

\usepackage{listing}

\begin{document}
\section*{1}
gegeben sei eine menge $A=\{x\in\mathbb{R}| irgentwas\}$ bestimmen sie inf(A) sup(A) max(A) und min(A) falls sie existieren.
\section*{2}
Zeigen sie dass irgendwas eine obere,untere Schranke f\"ur $A=\{x\in\mathbb{R}| irgentwas\}$ ist\\
\section*{3}
Bestimmen sie $\lim_{n\to\infty}$ von irgentwas%beispiele 
\section*{4}
Zeigen sie dass die Folge $a_n$ mit $a_n$ konvergent,divergent ist
\section*{5}
Zeigen sie dass die Folge $a_n$ mit $a_n$ eine nullfolge ist
\section*{6}
Benutzen sie das Wurzel,majoranten,minoranten,quotienten Kriterium um zu zeigen dass die reihe $\sum^{\infty}_{k=1}$ konvergent, absolut konvergent oder divergent ist 
\section*{7}
Berechnen sie den reihenwert von  $\sum^{\infty}_{k=1}$
\section*{8}
die folgen $a_n$ und $b_n$ sind konvergent,divergent zeigen oder widerlegen sie:\\
Die folge $(a_n+b_n)$ ist konvergent,divergent\\
skript: 4.1.17
\section*{9}
Die reihe $\sum^{\infty}_{k=1}a_k,a_k\in\mathbb{R}$ ist absolut konvergent, zeigen sie dass die reihe $\sum^{\infty}_{k=1}b_k=\sum^{\infty}_{k=1}a_k+\sum^{20}_{k=1}200$ absolut konvergent ist
\section*{10}
Zeigen sie dass die Funktion in $x=0$ stetig(differenzierbar) ist.
\section*{11}
Bestimmen sie alle lokalen und globalen extrema von \dots
\section*{12}
min-max
\section*{13}
taylor
\section*{14}
min-max
\section*{15}

\section*{16}
L’Hospital
\end{document}
