\textbf{Def. 1.5.8 (Max., Min., Sup., \& Inf.).} Es sei $(M,\preceq)$ eine Halbordnung und $X$ Teilmenge von $M$. Ein Element $a\in X$ heißt \{minimales / maximales\} Element in $X$, wenn für kein $x\in X$ gilt \{$x\prec a$ / $x \succ a$\}.

Ein Element $a \in X$ \{kleinstes / größtes\} Element in $X$, wenn für alle $x \in X$ gilt \{$a \preceq x$ / $a \succeq x$\}.

Ein Element $b \in X$ \{untere / obere\} Schranke von $X$, wenn für alle $x \in X$ gilt \{$b \preceq x$ / $b \succeq x$\}.

Ein Element $ b \in M$ heißt \{Infimum / Supremum\} von $X$, wenn $b$ \{größtes / kleinstes\} Element der Menge aller \{unteren / oberen\} Schranken von $X$ ist.

